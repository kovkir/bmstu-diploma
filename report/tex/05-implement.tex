\chapter{Технологический раздел}

\section{Средства реализации ПО}

В качестве языка программирования был выбран \texttt{Python} \cite{Python}. Это обусловлено наличием опыта работы с выбранным языком. Также для \texttt{Python} существует большое количество библиотек и документация на русском языке, а сам язык поддерживает объектно-ориентированную парадигму программирования.

При создании графического интерфейса для программного обеспечения была использована библиотека \texttt{tkinter} \cite{Tkinter}. Она является кроссплатформенной и включена в стандартную библиотеку языка \texttt{Python} в виде отдельного модуля.

Для визуализации работы методов разбиения использовалась библиотека \texttt{matplotlib} \cite{Matplotlib} с модулем \texttt{matplotlib.pyplot} \cite{Pyplot}. Для графического представления бинарного дерева, полученного в результате работы агломератиного подхода иерархического метода кластеризации, строилась дендрограмма с помощью модуля \texttt{cluster.hierarchy} \cite{ClusterHierarchy} библиотеки \texttt{scipy} \cite{SciPy}. Также для построения таблиц с результатами разбиения использовалась библиотека \texttt{prettytable} \cite{Prettytable}.

\section{Формат входных и выходных данных}

На вход программа получает файл в формате \texttt{csv}, содержащий данные как с категориальными, так и с числовые признаками. Перед началом разбиения необходимо знать номера столбцов с количественными параметрами, а также диапазоны их значений. Возможность выбирать входной файл через графический интерфейс не предусмотрена. Пользователь на вход подает следующие параметры:
\begin{itemize}
    \item количество итоговых кластеров;
    \item количество обрабатываемых объектов;
    \item количество прогонов для метода разбиения k-прототипов при сравнении методов.
\end{itemize}
Число обрабатываемых объетов не должно превышать количество строк в входном файле, число итоговых кластеров должно быть не больше количества объетов. Все входные параметры должны являться натуральными числами.

На выходе будет построена таблица с результатами выбранного метода разбиения. Например, для гибридного метода будет выведена таблица полученных кластеров с их центрами и номерами входящих в них объектов. Также будет построен график, визуализирующий результаты разбиения.
При проведении оценки качества кластеризации на выходе получим график сравнения методов, а также таблицу с вычисленными коэффициентами для каждого случая сравнения.

\section{Реализация гибридного метода разбиения}

Реализация гибридного метода кластеризации будет состоять из четырех основных этапов:
\begin{enumerate}
    \item Построение матрицы несходства.
    \item Применение агломеративного подхода иерархической кластеризации.
    \item Получение списка номеров объектов, входящих в каждый кластер.
    \item Уточнение принадлежности элементов к кластерам с помощью метода кластеризации k-прототипов.
\end{enumerate}

На вход разрабатываемый метод получает массив объектов с числовыми и категориальными признаками, а также итоговое количество кластеров.
Реализация класса \texttt{HybridClusterization} для гибридного метода разбиения данных на основе агломеративного подхода иерархической кластеризации представлена в листинге \ref{lst:hybridClusterization}.

На первом этапе необходимо построить матрицу несходства, которая необходима для определения степени различия между объектами и подается на вход следующему этапу кластеризации. Для определения расстоняния между элементами матрицы используется расстояние Говера. Стоит отметить, что матрица симметрична относительно главной диагонали, поэтому достаточно построить только верхнюю ее часть, нижнюю же можно будет заполнить отображением.
В листинге \ref{lst:distance} представлена реализация класса \texttt{Distance} для построения матрицы несходства и вычисления расстояния Говера между объектами.

На втором этапе матрица несходства передается в алгоритм агломератиного подхода иерархической кластеризации. В результате работы которого уже будут получены кластеры в виде списка узлов неполного бинарного дерева. В качестве критерия связи групп будет использована полная связь, то есть расстояние между кластерами будет считаться как самое длинное расстояние между двумя точками в каждом кластере. Данный этап считается основным, от результатов его работы во многом будет зависеть результат разрабатываемого метода.
Реализация класса \texttt{HAClusterization} для иерархической части гибридного метода кластеризации представлена в листинге \ref{lst:haClusterization}.

Третий этап считается переходным от второго к четвертому. Он необходим для преобразования узлов неполного бинарного дерева в списки номеров объектов, входящих в каждый кластер.

Четвертый этап кластеризации отвечает за уточнение полученных ранее групп. Для вычисления расстояния от объектов до центроидов (центров кластеров) также будет использовано расстояние Говера. Данный этап является последним в разрабатываемом алгоритме. В итоге гибридный метод разбиения вернет центроиды и списки кластеров с номерами входящих в них объектов.
Реализация класса \texttt{KPrototypesClusterization} для уточнения принадлежности элементов кластерам с помощью метода кластеризации центроидного типа представлена в листинге \ref{lst:kPrototypesClusterization}.

Чтобы не перегружать предоставленные листинги, из всех реализованных классов были убраны методы и переменные, отвечающие за графическое отображение результатов разбиения.
\clearpage

\section{Результаты работы ПО}

Взаимодействие пользователя с ПО осуществляется с помощью графического интерфейса (рисунок \ref{img:interface}), в котором предусмотрена возможность изменения количества обрабатываемых объектов, итогового числа кластеров, а также количества прогонов для k-прототипов при сравнении методов. При кластеризации данных пользователю предоставляется возможность выбирать один из трех реализованных алгоритмов разбиения. Также в ПО предусмотрена возможность оценки качества кластеризации с помощью метода оценки силуэтов и метода локтя.
\imgs{interface}{h!}{0.6}{Графический интерфейс разработанного ПО}

\clearpage
Основной особенностью иерархических методов кластеризации являются возвращаемые ими значения. В отличие от центроидных методов, которые возвращают списки образованных кластеров с их центрами и номерами входящих в них объектов, иерархические возвращают бинарное дерево кластеров, для визуализации которого используют дендрограмму. Результат работы агломеративного подхода иерархической кластеризации представлен на рисунке \ref{img:haRes}.
\imgs{haRes}{h!}{0.45}{Дендрограмма агломеративного подхода иерархической кластеризации}

По полученной дендрограмме можно сделать вывод о том, что для кластеризации использовались данные, находящиеся друг от друга на расстоянии не менее 0.33.

Для центроидных методов разбиения возникла проблема визуализации результатов работы. Так как кластеризирумые данные содержат числовые и категориальные признаки, их нельзя представить в виде точек в $n$-мерном пространстве. Поэтому было принято решение продемонстрировать расстояния от объектов до центров их кластеров. Чем больше объектов находится на меньшем расстоянии до центра, тем лучше.

\clearpage
На рисунках \ref{img:kPrototypesRes30}--\ref{img:kPrototypesRes60} представлены результаты работы метода разбиения k-прототипов для 30 и 60 итоговых кластеров.
\imgs{kPrototypesRes30}{h!}{0.46}{Результаты кластеризации методом центроидного типа k-прототипов для 30 кластеров}
\imgs{kPrototypesRes60}{h!}{0.46}{Результаты кластеризации методом центроидного типа k-прототипов для 60 кластеров}

\clearpage
На рисунках \ref{img:hybridRes30}--\ref{img:hybridRes60} представлены результаты работы разработанного гибридный
метода разбиения для 30 и 60 итоговых кластеров.
\imgs{hybridRes30}{h!}{0.46}{Результаты кластеризации гибридным методом для 30 кластеров}
\imgs{hybridRes60}{h!}{0.46}{Результаты кластеризации гибридным методом для 60 кластеров}

\clearpage
\section*{Вывод}

В данном разделе были рассмотрены средства реализации ПО, описан формат входных и выходных данных, реализован гибридный метод разбиения данных, а также приведены результаты работы ПО. 

Метод разбиения k-прототипов и гибридный метод относятся к центроидным алгоритмам разбиения. Однако результаты кластеризации обоих методов для 60 кластеров сильно отличаюся. Гибридный метод показывает лучшие результаты, так как объекты находятся ближе к центрам кластеров, чем у k-прототипов (при проведении оценки качества кластеризации методом локтя среднее внутрикластерное расстояние для гибридного метода составило 0.1, в то время как для k-прототипов --- 0.17). Это связано с рандомной инициализацией начальных центроидов в методе k-прототипов. В гибридном алгоритме центроиды вычисляются на основе кластеров, полученных в результате работы иерархической части метода.
