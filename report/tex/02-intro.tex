\maketableofcontents

\intro

Разбиение больших массивов категориальных данных является актуальной задачей в системах анализа данных в различных областях, таких как маркетинг, медицина, наука и другие. Цель кластеризации категориальных данных состоит в обнаружении скрытых закономерностей и разделении данных на группы, которые имеют схожие характеристики. Разбиение может быть использовано для ряда задач, таких как сегментация клиентской базы, предсказание потребительского поведения, снижение рисков, анализ и классификация текстовых документов.

В отличие от числовых данных, для категориальных не существует естественной метрики, которая может быть использована в качестве меры расстояния между объектами. Это делает кластеризацию категориальных данных более сложной задачей по сравнению с кластеризацией числовых данных.

Целью выпускной квалификационной работы является разработка метода разбиения категориальных данных на основе агломеративного подхода иерархической кластеризации.

Для достижения поставленной цели необходимо выполнить следующие задачи:

\begin{itemize}
    \item описать существующие методы кластеризации данных и сравнить их по выделенным критериям;
    \item описать существующие критерии связи кластеров в иерархическом методе разбиения данных;
    \item рассмотреть существующие меры расстояний между объектами и провести их сравнение;
    \item описать методы оценки качества кластеризации;
    \item разработать метод разбиения данных на основе агломеративного подхода иерархической кластеризации;
    \item разработать программное обеспечение для демонстрации работы созданного метода;
    \item провести сравнение разработанного метода разбиения с аналогами с помощью существующих методов оценки качества кластеризации.
\end{itemize}
