\chapter{Конструкторский раздел}

\section{Требования к разрабатываемому методу разбиения данных}

Для гибридного метода разбиения на основе агломеративного подхода иерархической кластеризации были выдвинуты следующие требования.
\begin{itemize}
    \item Разрабатываемый метод должен уметь работать с числовыми параметрами помимо категориальных, так как качественные признаки могут присутствовать вместе с количественными. Такое допущение позволит использовать данный метод для решении большего числа задач.
    \item Разрабатываемый алгоритм кластеризации на вход должен получать массив объектов с числовыми и категориальными признаками, а также итоговое количество кластеров.
    \item В результате работы гибридный метод разбиения должен возвращать два списка: центры образованных класстеров и сами кластеры с номерами входящих в них объектов.
\end{itemize}

\section{Требования к разрабатываемому ПО}

Для демонстрации работы гибридного метода необходимо разработать ПО со следующими требованиями.
\begin{itemize}
    \item Взаимодействие пользователя с ПО должно осуществляться с помощью графического интерфейса.
    \item Необходмо предусмотреть возможность изменения количества обрабатываемых объектов.
    \item Необходмо предусмотреть возможность изменения итогового количества кластеров.
    \item Пользователь должен иметь возможность сравнения гибридного метода разбиения с базовыми, на основе которых он был разработан, с помощью методов оценки качества кластеризации (методов локтя и оценки силуэтов).
\end{itemize}

\section{Проектирование метода разбиения данных}

При создании любого метода разбиения данных необходимо определить, каким образом будет определяться схожесть между объектами. Также, при использовании иерархического подхода в класстеризации, необходимо 
выбрать критерий связи кластеров.

Для разрабатываемого гибридного метода разбиения на основе агломеративного подхода иерархической кластеризации были выбраны следующие метрики.
\begin{itemize}
    \item В качестве критерия связи кластеров в иерархической части разрабатываемого метода должна быть использована полная связь, так как она обеспечивает более однородные и компактные кластеры, чем одиночная или средняя.
    \item В качестве меры расстояний при вычислении матрицы несходства должно быть использовано расстояние Говера, так как оно единственное из рассмотренных подходит для работы как с числовыми, так и с категориальными данными. При вычислении данного расстоняния будем считать, что отсутствующих значений в параметрах не существует.
\end{itemize}

После применения иерархической кластеризации и преобразования полученных данных (блоки A2 и А3 на рисунке \ref{img:idef0_a1}) необходимо уточнить принадлежность элементов кластерам с помощью метода кластеризации центроидного типа (блок A4 на рисунке \ref{img:idef0_a1}). В качестве такого уточняющего метода было выбрано разбиение k-прототипов, так как оно единственное из методов кластеризации центроидного типа способно обрабатывать данные, содержащие как категориальные, так и числовые признаки.
\clearpage

\section{Схемы разрабатываемого гибридного метода кластеризации}

Схема гибридного метода кластеризации представлена на рисунке \ref{img:hybrid}. Она состоит из четырех основных пунктов, три из которых далее будут рассмотрены более подробно.
\imgs{hybrid}{h!}{1}{Схема гибридного метода кластеризации}

\clearpage
Схема нахождения матрицы несходства представлена на рисунке \ref{img:dissimilarityMatrix}. Данная матрица показывает степень различия между объектами. Для определения расстоняния между элементами матрицы используется расстояние Говера.
\imgs{dissimilarityMatrix}{h!}{0.92}{Схема нахождения матрицы несходства}

\clearpage
Схема нахождения расстоняния Говера между двумя элементами с множеством признаков представлена на рисунке \ref{img:goverDistance}. Данное расстояние позволяет определить степень различия между объектами. Чем сильнее они отличаются друг от друга, тем ближе значение к 1.
\imgs{goverDistance}{h!}{0.73}{Схема нахождения расстоняния Говера}

\clearpage
Схема иерархической части гибридного метода кластеризации представлена на рисунке \ref{img:ha}. Это основная часть алгоритма, в результате которой уже будут получены кластеры в виде списка узлов неполного бинарного дерева.
\imgs{ha}{h!}{1}{Схема иерархической части гибридного метода кластеризации}

\clearpage
Схема центроидной части гибридного метода кластеризации представлена на рисунке \ref{img:kPrototypes}. Это последний этап гибридной кластеризации, в результате которого будет уточнена принадлежность элементов к кластерам.
\imgs{kPrototypes}{h!}{0.98}{Схема центроидной части гибридного метода кластеризации}

\clearpage
\section*{Вывод}

В данном разделе были предъявлены требования к разрабатываемому методу разбиения данных и к разрабатываемому ПО. Было произведено проектирование метода кластеризации. В качестве критерия связи кластеров в иерархической части метода была выбрана полная связь. Для вычисления матрицы несходства в качестве меры расстояний было выбрано расстояние Говера. Кроме того, в данном разделе были построены схемы для реализации гибридного метода разбиения.
