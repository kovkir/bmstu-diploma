\chapter{Исследовательский раздел}

\section{Применение методов оценки качества кластеризации}

\subsection{Метод локтя}

Данный метод оценки качества кластеризации применяется в тех случаях, когда важна компактность кластеров. Он оценивает среднее расстояние между объектами внутри групп. Метод локтя используют для определения оптимального количества кластеров путем нахождения на графике точки, после которой уменьшение среднего расстояния между элементами в группе будет не так заметно. 

Реализация данного метода оценки качества кластеризации представлена в листинге \ref{lst:testElbow}.
На рисунке \ref{img:testElbow} показаны результаты сравнения трех реализованных методов разбиения.
\imgs{testElbow}{h!}{0.45}{Сравнение методов разбиения с помощью метода локтя}

Результаты рассмотренного метода оценки качества разбиения приведены в таблице \ref{tbl:test_elbow}. В ней указаны только те строки, значения которых использовались в анализе результатов сравнения. Данная таблица показывает средние расстоняния между элементами в кластерах, образованных разными методами.
\begin{table}[H]
    \centering
	\caption{Результаты проведения оценки качества разбиения методом локтя}
    \label{tbl:test_elbow}
	\begin{tabular}{|c|c|c|c|}
        \hline
        \textbf{Кол-во кластеров} & \textbf{Иерархический} & \textbf{K-прототипов} & \textbf{Гибридный} \\ \hline
        11     &     0.483     &    0.552     &   0.537  \\ \hline
        12     &     0.494     &    0.531     &   0.484  \\ \hline
        13     &      0.49     &     0.53     &   0.443  \\ \hline
        26     &     0.463     &    0.345     &   0.287  \\ \hline
        27     &     0.471     &    0.347     &   0.276  \\ \hline
        28     &     0.454     &    0.344     &   0.283  \\ \hline
        30     &     0.409     &    0.311     &   0.291  \\ \hline
        33     &     0.311     &    0.295     &   0.254  \\ \hline
        34     &     0.282     &    0.292     &   0.227  \\ \hline
        35     &     0.255     &    0.291     &    0.22  \\ \hline
        36     &      0.23     &    0.291     &   0.214  \\ \hline
        37     &     0.206     &    0.269     &    0.19  \\ \hline
        38     &     0.183     &    0.284     &   0.183  \\ \hline
        39     &     0.161     &     0.26     &   0.161  \\ \hline
        40     &     0.156     &    0.258     &   0.157  \\ \hline
        41     &     0.151     &    0.264     &   0.152  \\ \hline
        42     &     0.155     &    0.227     &   0.156  \\ \hline
        60     &     0.099     &    0.166     &   0.099  \\ \hline
    \end{tabular}
\end{table}

Гибридный метод показал лучшие результаты при итоговом количестве кластеров от 12 до 37. Максимальная разница наблюдалась при 27 группах (у гибридного метода среднее внутрикластерное расстояние было в 1.26 раз меньше, чем у k-прототипов и в 1.71 раз меньше, чем у иерархического). При дальнейшем увеличении групп среднее расстояние для иерархического и гибридного методов было практически одинаковым. Это связано с рандомной инициализацией центров кластеров в методе k-прототипов, из-за чего центроидный метод показывал худшие результаты, чем иерархический, начиная с 34 групп (в 1.65 раз хуже при 40 и в 1.68 раз хуже при 60). Из-за этого уточнение значений на последнем этапе гибридного метода стало бесполезным. Отсюда следует, что гибридный метод разбиения данных надо применять в тех случаях, когда метод k-прототипов показывает лучшие результаты в сравнении с иерархическим. 

По результатам оценки качества кластеризации методом локтя можно сделать вывод о том, что оптимальным количеством групп будет 39, так как далее среднее расстояние будет уменьшаться незначительно (в 1.03 раза для 40 и 41 кластеров, тогда как для 38 и 39 кластеров оно уменьшилось в 1.13 и в 1.14 раз соответственно). Оценивались результаты иерархического метода разбиения, так как на них лучше всего видна точка <<локтя>>. 

\subsection{Метод оценки силуэтов}

Данный метод оценки качества кластеризации применяется в тех случаях, когда важно среднее расстояние между группами. Для этого вычисляется коэффициент силуэта в диапазоне $[-1, 1]$, определяющий, насколько близко каждая точка внутри одного кластера расположена к точкам ближайшей соседней группы. Чем ближе значение к 1, тем лучше кластеризированы объекты. 

Реализация данного метода оценки качества кластеризации представлена в листинге \ref{lst:testSilhouettes}.
На рисунке \ref{img:testSilhouettes} показаны результаты сравнения трех реализованных методов разбиения.
\imgs{testSilhouettes}{h!}{0.43}{Сравнение методов разбиения с помощью метода оценки силуэтов}
\clearpage

Результаты метода оценки силуэтов приведены в таблице \ref{tbl:test_silhouettes}. В ней указаны только те строки, значения которых использовались в анализе результатов сравнения. Данная таблица показывает средние значения коэффициентов силуэта для элементов в кластерах, образованных разными методами.
\begin{table}[H]
    \centering
	\caption{Результаты проведения оценки качества разбиения методом оценки силуэтов}
    \label{tbl:test_silhouettes}
	\begin{tabular}{|c|c|c|c|}
        \hline
        \textbf{Кол-во кластеров} & \textbf{Иерархический} & \textbf{K-прототипов} & \textbf{Гибридный} \\ \hline
        17    &    0.246    &   0.165    &  0.241  \\ \hline
        18    &    0.247    &   0.174    &  0.254  \\ \hline
        19    &    0.246    &   0.164    &  0.268  \\ \hline
        30    &    0.343    &   0.257    &  0.401  \\ \hline
        36    &    0.491    &    0.25    &  0.499  \\ \hline
        37    &    0.516    &   0.295    &  0.524  \\ \hline
        38    &     0.54    &   0.273    &  0.539  \\ \hline
        39    &    0.565    &   0.285    &  0.564  \\ \hline
        40    &    0.555    &   0.283    &  0.554  \\ \hline
        52    &    0.561    &   0.427    &  0.561  \\ \hline
        53    &     0.57    &   0.385    &   0.57  \\ \hline
        54    &    0.581    &    0.4     &  0.581  \\ \hline
        60    &    0.609    &   0.417    &  0.609  \\ \hline
        80    &     1.0     &   0.588    &   1.0   \\ \hline
    \end{tabular}
\end{table}

Метод кластеризации k-прототипов показал наихудшие результаты в сравнение. Это связано с рандомной инициализацией центроидов, из-за чего некоторое кластеры образовывались близко друг к другу. Также принцип случайности при инициализации приводил к тому, что элементы неравномерно распределялись по группам, некоторые из которых вообще оставались пустыми. В отличие от центроидного метода, иерархический и гибридный изначально каждый объект закрепляли за своим кластером. Такой подход гарантировал, что пустых групп оставаться не будет. 

Гибридный метод показал наилучшие результаты в промежутке от 18 до 37 итоговых кластеров. Максимальная разница наблюдалась при 27 группах (у гибридного метода коэффициент силуэта был в 1.27 раз больше, чем у иерархического и в 2.02 раза больше, чем у k-прототипов). Стоит отметить, что когда число кластеров совпало с числом объектов, то среднее значение коэффициента силуэта для гибридного и иерархического методов равнялось 1. Это показывает, что каждый объект находился в своем кластере, и что пустых групп не было. 

По результатам оценки качества кластеризации методом оценки силуэтов можно сделать вывод о том, что оптимальным количеством групп будет 39, так как в данной точке на графике наблюдается экстремум, после которого коэффициент силуэта вновь начинает расти относительно точки максимума только при 53 кластерах.
В данной точке экстремума среднее значение коэффициента силуэта для гибридного и иерархического методов будет в 1.98 раз превышать показатель метода k-прототипов.

\section*{Выводы}

В данном разделе проводилось сравнение реализоваанных алгоритмов разбиения с помощью существующих методов оценки качества кластеризации: метода локтя, метода оценки силуэтов.

Метод локтя показывает среднее расстояния между объектами внутри групп. 
Когда в наборе данных существует несколько объектов, идентичных относительно друг друга по вычисленному расстоянию (сами объекты необязательно должны быть одинаковыми), то центроидный метод разбиения будет возвращать более компактные кластеры, чем иерархический. 
Именно в таких случаях для достижения максимальной компактности лучше всего использовать гибридный метод кластеризации, так как он использует центроиды для уточнения полученных значений и не применяет рандомную инициализацию первичных центров кластеров, из-за чего является более предпочтительным в сравнении с иерархическим и k-прототипов.

В результате использования метода оценки силуэтов было выяснено, что максимально разделенные кластеры получаются в результате работы иерархического и гибридного методов разбиения. Гибридный также будет предпочтительнее (средние значения коэффициентов силуэта будут больше), когда в наборе данных существует несколько обособленных объектов, равноудаленных друг от друга.
