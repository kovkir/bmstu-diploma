\conclusion

В ходе выполнения выпускной квалификационной работы надо было разработать метод разбиения категориальных данных на основе агломеративного подхода иерархической кластеризации. Поставленная цель была достигнута. Также были выполнены следующие задачи.
\begin{enumerate}
    \item Описаны существующие методы кластеризации данных (иерархический, k-средних, k-режимов, k-прототипов, C-средних, DBSCAN, МПД) и проведено их сравнение по выделенным критериям. Для уточнения результатов иерархического разбиения в гибридном методе был выбран центроидный метод k-прототипов.
    \item Описаны существующие критерии связи кластеров (одиночная, полная, средняя) в иерархическом методе разбиения данных. При создании гибридного метода была использована полная связь, которая вычисляет расстояние между двумя кластерами как самое длинное расстояние между двумя точками в каждой группе. Полная связь была выбрана, так как она обеспечивает более компактные кластеры, чем одиночная или средняя.
    \item Рассмотрены существующие меры расстояний между объектами (Евклидово расстояние, квадрат Евклидово расстояния, расстояние городских кварталов, расстояние Чебышева, расстояние Минковского, степенное расстояние, расстояние Хэмминга, расстояние Говера) и проведено их сравнение по возможности обрабатывать данные, содержащие категориальные и числовые признаки. Для гибридного метода кластеризации было выбрано расстояние Говера, так как оно едиственное из рассмотренных может обрабатывать оба типа признаков.
    \item Описаны методы оценки качества кластеризации (метод локтя, метод оценки силуэтов). Первый используют, когда важна компактность кластеров, второй, когда важно среднее расстояние между группами.
    \item Разработан метод разбиения данных на основе агломеративного подхода иерархической кластеризации.
    \item Разработано программное обеспечение для демонстрации работы созданного метода.
    \item Проведено сравнение разработанного метода разбиения с аналогами с помощью существующих методов оценки качества кластеризации. C помощью метода локтя было выяснено, что когда в наборе данных существует несколько обособленных объектов, равноудаленных друг от друга, то для достижения максимальной компактности лучше всего использовать гибридный метод разбиения. В результате использования метода оценки силуэтов было показано, что максимально разделенные кластеры получаются в результате работы иерархического и гибридного методов разбиения.
\end{enumerate}
