\addcontentsline{toc}{chapter}{РЕФЕРАТ}
\chapter*{РЕФЕРАТ}

Расчетно–пояснительная записка к выпускной квалификационной работе содержит \begin{NoHyper}\pageref{LastPage}\end{NoHyper} страниц, \totfig~иллюстраций, \tottab~таблиц, 30 источников, 3 приложения. 

В данной работе представлена разработка метода разбиения категориальных данных на основе агломеративного подхода иерархической кластеризации.

Описаны существующие методы кластеризации данных и проведено их сравнение по выделенным критериям. Описаны существующие критерии связи кластеров в иерархическом методе разбиения данных. Рассмотрены существующие меры расстояний между объектами и проведено их сравнение. Описаны методы оценки качества кластеризации. Разработан метод разбиения данных на основе агломеративного подхода иерархической кластеризации. Разработано программное обеспечение для демонстрации работы созданного метода. Проведено сравнение разработанного метода разбиения с аналогами с помощью существующих методов оценки качества кластеризации.

Ключеые слова: кластеризация, кластер, разбиение, иерархический метод, метод k-прототипов, категориальные данные, расстояние Говера, полная связь, метод локтя, метод оценки силуэтов.
